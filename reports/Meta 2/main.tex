\documentclass{article}
\usepackage[utf8]{inputenc}

\title{Projeto Multimédia - FlyBeat}
\author{André Batista \thanks{nº2012137523, amccbaptista@gmail.com} \and João L. Cardoso \thanks{nº2011151968, jaliborc@gmail.com} \and Rui F. Casaleiro \thanks{nº2012139327, rjfcasaleiro@gmail.com}}
\date{LEI - Meta 2}

\usepackage{natbib}
\usepackage{graphicx}

\begin{document}
\maketitle
\section{Resumo do Trabalho}
Neste trabalho iremos desenvolver um jogo de aviação inspirado no jogo Audiosurf. O jogador conduzirá um avião através de um túnel tridimensional evitando colidir com as paredes do mesmo. Haverá um número limite de colisões que um jogador pode fazer.

O formato do túnel dependerá da análise de um ficheiro de música à escolha do utilizador, que servirá também como música de fundo para o jogo. Para esse efeito, analisaremos a melodia e a energia da mesma. Consideramos também a possibilidade da música poder criar turbulência na condução do avião em momentos-chave, e de ter efeitos sonoros dependendo do desempenho do jogador.

Um dos aspectos mais distintivos do jogo será possibilidade de controlar o
avião utilizando os sensores de rotação de um smartphone Android. Como alternativa o avião poderá também ser controlado utilizando o teclado, mas iremos desenhar o jogo com o smartphone em mente.

Temos também em mente ter, no menu inical, um registo do progresso do jogador em cada música - a percentagem da música à qual conseguiu "sobreviver" e com quantas falhas.

\section{Software Utilizado}
\subsection{Tecnologias}
\begin{itemize}
\item Away3D (Baseado em OpenGL)
\item Networking (Sockets TCP)
\item Android Sensors (Gyroscope - sensor de rotação)
\item Análise musical recorrendo a séries de Fourier
\end{itemize}

\subsection{Ferramentas}
\begin{itemize}
\item Adobe Flash Builer 4.7 (Desenvolvimento de código)
\item Flex 4.6 (Necessário para a library Away3D)
\item Android Developer Tools 22.3 (Desenvolvimento do controlo remoto)
\item Maya (Desenho de modelos 3D)
\item Adobe Fireworks CS5 (Planeamento de interfaces)
\item Prefab3D (Integração dos modelos produzidos na library Away3D)
\item Git (Desenvolvimento cooperativo)
\item Draw.IO (Desenho de esquemas auxiliares)
\end{itemize}

\section{Diagrama de Navegação}


\section{Layout da Interface}

\section{Preview de Gameplay}


\section{Diagrama de Classes}

\section{Diagrama de Gant}

\end{document}